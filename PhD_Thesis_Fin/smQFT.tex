\subsection{From the quantum field theory window }
\label{sec:StandardModelQFT}
In particle mechanics, a Lagrangian\footnote{L=T-V, T is the kinetic energy of the particle in a potential V.} is a function of the coordinates, and their time derivatives. In field theory a Lagrangian density is used and it is a function of the fields, $\phi_i$ and their position and time derivatives, $\partial_{\mu}\phi_i $. In relativistic theory, space and time coordinates are treated on equal footing. The Lagrangian plays an important role in physics because it encodes both the dynamics and the symmetries of the theory. Given a Lagrangian, the equation of motion from the Euler-Lagrange equations can be derived by considering the least action principle, i.e requiring that the variation of action is zero\footnote{$\delta S=0$, where $S=\int L dt$.}. Then, the Euler-Lagrange equation can be written as:
\begin{eqnarray}
\label{EuLag}
% \nonumber to remove numbering (before each equation)
	{ \partial_{\mu}\frac{\partial L}{\partial(\partial_{\mu}\phi_i)}} = {\frac{\partial L}{\partial\phi_i}} .
	%\,\frac{\patial}{\partial(\partial_{\mu}\phi_i)}
\end{eqnarray}
The \acrshort{sm} is a relativistic \acrfull{qft}. Its Lagrangian is built on a global Poincar\'{e} symmetry, which implies symmetry under translations, rotations and Lorentz boosts. According to Noether\textquoteright s theorem~\cite{Noether}, each continuous symmetry is accompanied by a conservation law. The Poincar\'{e}‬ symmetry implies the conservation to conservation of energy, momentum, and angular momentum. The gauge group of the SM, which is a local symmetry, is described as:  
\begin{eqnarray}
\label{SMsym}
% \nonumber to remove numbering (before each equation)
	{SU(3)_C \otimes SU(2)_L \otimes U(1)_Y},
\end{eqnarray}
where $SU(3)_C$, $SU(2)_L$ and $U(1)_Y$ are representing the gauge groups of the strong, weak and electromagnetic forces respectively. The subscript $C$ refers color, $L$ refers to left-handedness, and $Y$ refers hypercharge. 
The conserved quantities, which correspond to the symmetry in Eq.~\ref{SMsym}, are color charge, charge, and weak hypercharge.\\
The rank of the group, i.e. the number of generators of the fields is at the same time the number of mediators, gauge bosons, of the corresponding vector field.
For instance, the $SU(3)_C$ group has eight\footnote{$N_c^2-1$, where $N_c=3$} generators thus it has eight vector fields which are called the gluon fields ($G_{\mu}$). Following the same argument, the $SU(2)_L$  group has three vector fields, ($W_{\mu}^1$, $ W_{\mu}^2$, and $ W_{\mu}^3$) and the $U(1)_Y$ group has only one vector field, ($B_{\mu}$). \\
In addition, the \acrshort{ewsb}, or in other words the Higgs mechanism, leads to an additional scalar field, which will be denoted as $\phi$ in upcoming equations. 
The SM Lagrangian resides two components: $L_{QCD}$ and $L_{EWK}$. The first one is for strong interaction while the latter explains electroweak interaction including interaction with the Higgs boson. 
%\newpage
\begin{itemize}
  \item \textbf{Strong interaction}
\end{itemize}
As mentioned in the previous section, the theory that aims to model strong force is \acrshort{qcd}. The \acrshort{qcd} Lagrangian is given by:
\begin{eqnarray}
\label{LagQCD}
{L_{QCD}} = {\bar{\Psi}(i\gamma^{\mu}\partial_{\mu}-m)\Psi + g_s\bar{\Psi}T_aG_{\mu}^a\gamma^{\mu}\Psi - \frac{1}{4}G_{\mu\nu}^aG^{\mu\nu}_a}.
\end{eqnarray}
where $\Psi$ are the quark fields, $\gamma^{\mu}$ are the Dirac matrices, $g_s$ is the strong coupling constant, $T_a$ are Gell-Mann matrices and $G_{\mu\nu}^a \equiv \partial_{\mu}G_{\nu}^a - \partial_{\nu}G_{\mu}^a + g_sf_{abc}G_{\mu}^bG_{\nu}^c$\footnote{$f_{abc}$ are the structure constants}.
When constructing the Lagrangian, a covariant derivative is introduced such that the kinetic term stays invariant under gauge transformations. In general, the form of the covariant derivative is $D_{\mu}=\partial_{\mu}-igX_{\mu}$ where with the $g$ coefficient fermion interacts with the X boson. Concretely, the kinetic terms are read as $\bar{\Psi}\gamma^{\mu}D_{\mu}\Psi$, and the covariant derivative for \acrshort{qcd} is defined as: $D_{\mu}=\partial_{\mu}-ig_sT_aG_{\mu}^a$.
%\newpage
\begin{itemize}
  \item \textbf{Electroweak interaction}
\end{itemize}

The electroweak Lagrangian, $L_{EWK}$, can be written as a sum of four contributions:
\begin{eqnarray}
\label{LagEWK1}
{L_{EWK} = L_{Gauge}+L_{Fermion}+L_{Higgs}+L_{Yukawa}}.
\end{eqnarray}
The covariant derivative that leaves this Lagrangian invariant under gauge transformations is described as: $D_{\mu}=\partial_{\mu}-igW_{\mu}^a\tau_a - ig'B_{\mu}Y_W$ where $g$ and $g'$ are the the gauge couplings of the $S U(2)_L$ and $U(1)_Y$ respectively. The $\tau_a$ denote the Pauli matrices. The weak hypercharge is denoted by $Y_W$ and it is defined as $2(Q-I_3)$. The electric charge is denoted by $Q$ and $I_3$ is the third component of the weak isospin.
The first term in Eq.~\ref{LagEWK1} is defining the interaction among the gauge bosons, and it can be written as:
\begin{eqnarray}
\label{Laggauge1}
{L_{Gauge} = -\frac{1}{4}W_{\mu\nu}^aW^{\mu\nu}_a -\frac{1}{4}B_{\mu\nu}B^{\mu\nu} }
\end{eqnarray}
where $W_{\mu\nu}^a = \partial_{\mu}W_{\nu}^a - \partial_{\nu}W_{\mu}^a + g\epsilon^{abc}W_{\mu}^bW_{\nu}^c$,
 $B_{\mu\nu} = \partial_{\mu}B_{\nu} - \partial_{\nu}B_{\mu}$.
The second term in Eq.~\ref{LagEWK1} stands for the fermion kinetic term and fermion interactions with $SU(2)$ and $U(1)$ bosons:
\begin{eqnarray}
\label{Lagfermions1}
{L_{Fermion} = i\bar{\Psi}\gamma^{\mu}D_{\mu}\Psi}
\end{eqnarray}
The third term in the same equation is the Higgs Lagrangian, describing the Higgs field, its self-interaction and its interaction with the gauge bosons:
\begin{eqnarray}
\label{LagHiggs1}
{L_{Higgs} = |D_{\mu}\phi|^2 - \lambda(|\phi|^2 - \frac{\nu^2}{2})^2},
\end{eqnarray}
where $\lambda$ is the Higgs self-coupling strength. %, $\nu$ is the vacuum expectation value (VEV) and $\nu^2 > 0$.
According to BEH mechanism, there is a scalar potential, which permeates the whole universe: 
\begin{eqnarray}
\label{higgspot}
% \nonumber to remove numbering (before each equation)
	{V(\Phi)}= {m^2\Phi^{\dagger}\Phi+\lambda(\Phi^{\dagger}\Phi)^2},
\end{eqnarray}
with the Higgs field $\Phi$ with weak hypercharge $Y =1$, and a self-interacting SU(2) complex doublet in Eq.~\ref{higgsfield}\footnote{$\phi^0$:CP-even,$a^0$:CP-odd neutral component. $\phi^+$:complex charged component}.
\begin{eqnarray}
\label{higgsfield}
% \nonumber to remove numbering (before each equation)
	{\Phi}= {\frac{1}{\sqrt2}}{\sqrt2\phi^{+}  \choose \phi^0 + ia^0 }
\end{eqnarray}
In the case where $m^2$ in Eq.~\ref{higgspot} is positive, the potential acquires a ground state at origin. Then the theory is in the form of \acrshort{qed} with a massless photon and charged scalar field $\phi$ with a mass m. However, if $m^2<0$, the potential has then an infinite number of minima. The shape of the potential is generally referred as a “Mexican hat”. In this ground state, system has a broken symmetry and the Higgs field can be written as:
\begin{eqnarray}
\label{higgsfield_vacuum}
% \nonumber to remove numbering (before each equation)
	|\Psi|= \sqrt{\frac{-\mu^2}{2\lambda}}.
\end{eqnarray}
A direction choice of the Higgs field as: $\phi^+=0$, $a^0=0$ and $\phi^0=\sqrt{\frac{-\mu^2}{2\lambda}}=v$ ($v\equiv \acrshort{vev})$, results in three massless Goldstone bosons and one massive Higgs boson. Moreover, these Goldstone bosons disappear when gauge invariance is required. The fluctuation around the minimum $v$ can be written as :
\begin{eqnarray}
\label{higgsfield_vacuum_dir}
% \nonumber to remove numbering (before each equation)
	{\phi(x)}= \frac{1}{\sqrt{2}}{0 \choose v+h(x)}\,,
\end{eqnarray}
where the scalar field $h(x)$ portrays a physical Higgs boson. 
In the expanded version of the potential the coefficient of the $h^2$ term gives the Higgs boson mass:
\begin{eqnarray}
\label{higgsfield_vacuum_dir}
% \nonumber to remove numbering (before each equation)
	{M^2_H}= {2\lambda v^2 \,,\,M_H=\sqrt{2}|m|}.
\end{eqnarray}
%If the quadratic term is negative, the neutral element of the scalar doublet gains a non-zero \acrshort{vev} stimulating the  \cite{Olive:2016xmw}.
The final term in Eq.~\ref{LagEWK1}  is the Lagrangian of the Yukawa interaction between the Higgs field and the fermion fields (quarks and leptons). The $L_{Yukawa}$ produces fermion masses through \acrfull{ssb}. It can be written in the most general way~\cite{srednicki}:
\begin{eqnarray}
\label{LagYukawa1}
{L_{Yukawa} = -\epsilon^{ij}\phi_i\ell_{jI}y_{Ij}\bar{e}_j - \epsilon^{ij}\phi_iq_{\alpha jI}y'_{Ij}\bar{d}_{j}^\alpha - \phi^{\dagger i}q_{\alpha iI}y''_{Ij}\bar{u}_{j}^\alpha +h.c.}\,
\end{eqnarray}
where $y_{Ij}$, $y'_{Ij}$, $y''_{Ij}$ are complex $3\,\times\,3$ matrices, and the generation indices ($I=1,2,3$) are summed.
After \acrshort{ssb} the neutral and charged current interactions between fermions and gauge bosons can be derived from the $L_{EWK}$.  In order to extract the actual mass terms, a switch between basis with $W^a$, $B$ fields and a basis with mass eigenstates is desirable:
\begin{eqnarray}
\label{Zbos}
% \nonumber to remove numbering (before each equation)
	{ \gamma \choose Z } = { cos \theta_W \, \,\, \, sin \theta_W \choose -sin \theta_W \,\, cos \theta_W  } { B^0 \choose W^3 }
\end{eqnarray}
\begin{eqnarray}
\label{Wbos}
% \nonumber to remove numbering (before each equation)
	{W^{\pm}}= {W^1+iW^2} .
\end{eqnarray}
The $\theta_W $ term is the Weak mixing angle or Weinberg angle. This quantity is measured experimentally as well.
Rewriting the Lagrangian in Eq.~\ref{Laggauge1} in terms of the physical gauge bosons, the mass terms for massive bosons $V$ arise in the form of $\frac{1}{2}M^2_VV^2_{\mu}$. Using the coefficients masses of the charged and neutral bosons can be written as:
\begin{eqnarray}
\label{bos_masses}
% \nonumber to remove numbering (before each equation)
	{M_{W^{\pm}}}= {\frac{1}{2}vg} ,\nonumber\\
	{M_Z}={\frac{1}{2}v\sqrt{(g^2+g'^2)}}.
\end{eqnarray}
Since $g$ and $g′$ are free parameters, the \acrshort{sm} makes no absolute predictions for $M_W$ and $M_Z$. However, it was possible to set limits before the discovery of $W$, $Z$ bosons.
Considering the muon decay, using the relations in Eq.~\ref{bos_masses} and the Fermi constant given in the footnote\ref{fn:vev}, $v$ is found as 246 GeV.
Looking back at the mass term for the mass of the Higgs boson in Eq.~\ref{higgsfield_vacuum_dir}, since $\lambda$ is a free parameter, the mass of the Higgs boson is not predicted in the \acrshort{sm}.\\

%\newpage
