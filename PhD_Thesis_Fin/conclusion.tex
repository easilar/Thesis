\chapter*{Conclusion}
The Standard Model of particle physics continues to be a successful theory in explaining the phenomena observed in experiments, such as those at the LHC. However, the SM fails to elucidate several experimental facts that give rise to questions regarding the origin of our universe such as: What is Dark Matter and how can the quantum Higgs mass corrections be stabilized?\\
The supersymmetric extension of the SM is one of the most appealing beyond the SM theories that provide solutions to some of these questions. SUSY relates the SM fermions to bosonic superpartners and bosons to fermionic superpartners. This characteristic allows SUSY to reduce the divergent corrections on the Higgs mass.  Furthermore, the R-parity conserved SUSY models provide a cold-dark matter candidate in the form of the lightest SUSY particles. \\
In this thesis, a generic search for SUSY is performed in events with single lepton final states, multiple jets and none of them b-tagged. The sensitivity of the search is shown with a simplified model of SUSY, where each of the pair produced gluinos decays to neutralino, which is considered to be stable lightest supersymmetric particle, together with several light quark jets and one electron or muon. The two neutralinos and the neutrino coming from the leptonic W boson decay combine a large $\MET$. Moreover, the high multiplicity of jets leads to large hadronic activity. Therefore the model has high sensitivity in the tails of the kinematical distributions.  \\
The main search variable of the analysis is the azimuthal angle between the lepton and reconstructed $W$ boson, $\Delta\phi(W,\ell)$. The leading background processes tend towards low values of the angle while the expected signal events show a flat distribution, due to the large missing transverse energy contribution from LSP. Thus, the region with high (low) values of this angle is chosen to be signal (control) region. To further increase the sensitivity several signal rich search regions are defined, based on the number of jets, the hadronic scale ($\HT$), and the leptonic scale ($\LT$).\\
The Standard Model background is predicted with a robust data-driven approach, based on transfer factors $\Rcs$. The method uses control regions to estimate the normalization in signal regions and low jet multiplicity sidebands to obtain the signal to control region transfer factors, $\Rcs$. The main SM backgrounds from $\wJets$ and $\ttJets$ production are predicted separately, and a further separate estimation method is developed for QCD multi-jets events.  This method uses the polarization information of leptons from W boson decays which manifests itself in the $\rm L_P$ distributions. The predicted QCD multi-jets event counts are subtracted from the control regions. All of the methods are validated in simulation and data. \\
Various sources of systematic uncertainties on the background prediction are estimated. The largest contributions are from possible residual dependencies of $\Rcs$ on the jet multiplicity and potential mismodeling of the dileptonic fraction of $\ttbar$ events. Additionally, the systematic uncertainties coming from the simulated signal events are studied and their effects are propagated to the result. The largest uncertainties arise from jet energy correction uncertainties and potential mismodeling of pile up. Other uncertainties are at the 5\% level or lower.\\
Finally, the results of the data-based prediction is compared to data. The data is recorded by the CMS experiment during the 2016 run of proton-proton collisions of LHC at 13 TeV center of mass energy. The dataset corresponds to an integrated luminosity of 35.9 fb$^{-1}$. No deviation from the SM background is observed. Therefore, upper limits at 95\% CL on gluino production cross section are obtained in the context of the investigated simplified model, where the intermediate chargino mass in decay chain is taken to be half way between gluino and neutralino. As a result, gluino masses below 1.9 TeV are excluded for neutralino masses below 300 GeV. This corresponds to an improvement of about 500 GeV over the previous result \cite{SUS_16_005}.\\
After the 35.9 fb$^{-1}$ of the 13 TeV run, data still do not favor the SUSY models. The MSSM becomes more and more constrained. As the limits on the gluino mass approach to two TeV scale, natural SUSY becomes more restricted. Given the variety of possibilities, completely excluding SUSY is difficult. In the future, SUSY searches can be expanded towards long-lived particles with displaced vertex analyses, and there is still room for natural SUSY with compressed mass configurations. \\
Although no significant deviation from the Standard Model has been observed, the journey of particle physics is not yet finished. The unanswered questions from the observation of astrophysical phenomena remains as they are, as long as an evidence of a more universal model is not observed. \\
When viewed from this perspective, the results of the present work may look like a small step. However, even if small, it has increased our knowledge about nature, and, moreover, it is a step that is firmly rooted in data. As such, it will prevail.
%Over the last decades, the experimental searches have been performed under the guidance of theoretical physics methods. The success of this scientific path is proven with the Higgs boson discovery and the gravitational waves.
%It is also possible that a sudden unexpected discovery can illuminate the path of theoretical physics. 