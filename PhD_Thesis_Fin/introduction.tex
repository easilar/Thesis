%\begin{Introduction}
\chapter*{Introduction}
%FIXME: To be written in a nicer way, at the end...
%The new energy level of LHC allowed experimentalists to perform various searches.
%These searches have probbed a broad variety of possible theories that can explain how the universe is working, i.e
%from how objects are falling to how quarks compose protons.
%One of the most promissing ones was the Standard model (SM) of particle physics. As a result of comprehensive searches performed by the CMS and ATLAS experiments at LHC the discovery of Higgs Boson in 2012 completed the last missing part of the theory.
%Up to now, the SM has succesfully explained the physics observed in LHC.
%However, SM has several shortcomings.
%The lack of explanation of gravity and dark matter are some of the most important ones.
 %Moreover, the way SM is introducing the Higgs boson mass includes an unnatural fine tuning, so called hierarchy problem. 
%Therefore, the journey of particle physics has not ended. 
% Nowadays some of the most favourable searches to understand what is beyond the SM are searches for supe searches.  
%There are many extensions of the SM. One of the most favourable ones is Supersymmetry (SUSY). 
%%%%%%%%%%%
%The new energy level of LHC allowed experimentalists to perform various searches.
%These searches have probed a broad variety of theories that can possibly explain how the universe is working, i.e from how objects are falling to how protons compose of quarks and gluons.
%One of the most promising ones was the Standard model (SM) of particle physics. As a result of comprehensive searches performed by the CMS and ATLAS experiments at LHC, the discovery of Higgs Boson in 2012 completed the last missing part of the theory.
%Even though the SM has been tested many times until now, It predicts the quantifiable quantities of interactions between the properties of subatomic particles with great precision. Hence, the SM has successfully explained the physics observed in LHC.
%However, SM has several shortcomings.
%The lack of explanation of gravity and dark matter are some of the most important ones.
 %Moreover, the way SM is introducing the Higgs boson mass includes an unnatural fine-tuning, so called hierarchy problem. 
%Therefore, the particle physics continues to its journey to find solutions to these problems. 
%One of the most favorable ones was proposed as an extension of space-time symmetry associates the fermions and the bosons. The theory is meaningfully named as Supersymmetry (SUSY). 
%It is successfully solving the two of the problems mentioned above. It is providing a weakly interacting particle candidate for dark matter with involving a stable neutral supersymmetric particle. Moreover, the cancellations between bosonic and fermionic states result in a reduced size of quantum corrections. Hence, SUSY includes a solution to the hierarchy problem by itself.

%This thesis focuses on a search for Supersymmetry in events with a single lepton performed on proton-proton collisions at a center-of-mass energy of 13 TeV. The data is recorded by the CMS experiment during Run 2 of the LHC.
%A R-parity conserving simplified model describing the decay of pair-produced gluinos, superpartners of the Standard Model gluons is studied. Each gluino decays to two light quarks and an intermediate chargino, with the latter decaying to a W boson and a neutralino. In this model, the neutralino is considered to be the stable lightest supersymmetric particle. Hence, It is a sensible candidate of dark matter.
%The main search variable is the azimuthal angle between the lepton and four-vector sum of the missing energy and lepton. The angle for leading background processes tend towards low values while the expected signal events do not show dependence, due to the large missing transverse energy contribution from LSP. Thus, the region with high (low) values of this angle is chosen to be signal (control) region.  The Standard Model background is estimated with a data-driven approach using control regions where no signal contribution is expected. Low jet multiplicity sidebands are used to obtain signal to control region transfer factor.

The \acrfull{sm}~\cite{glashow,salam,weinberg,higgs1,higgs2} describes fundamental particles and their interactions through the electromagnetic, the weak and the strong force. The discovery of the entire SM particle content was completed~\cite{top1,top2,higgsexp1,higgsexp2,higgsexp3} with the various high-energy experiments, for example, the Tevatron at Fermilab between 1983 and 2011, the Large Electron-Positron Collider (LEP) at CERN between 1989 and 2000, and the \acrfull{lhc} at CERN starting its first scientific run in 2010. Up to now, results are consistent with the SM within uncertainties. However, results from the experiments such as Planck~\cite{DM5}, successor of COBE~\cite{DM4} and WMAP~\cite{DM3}, suggest a cold dark matter candidate which the SM fails to explain. Many extensions of the SM provide solutions to these problems, and \acrfull{susy}~\cite{Psusy1,Psusy2,Psusy3,Psusy4,Psusy5,Psusy6,Psusy7,Psusy8} is among the most promising candidates. It relates fermionic states to bosonic states.\\
A generic search for SUSY in events with a single electron or muon is performed using proton-proton collisions at a center-of-mass energy of 13 TeV. The data were recorded by the \acrfull{cms}~\cite{CMS_exp} experiment during the 2016 Run of the LHC, corresponding to an integrated luminosity of 35.9 fb$^{-1}$. The example signal model is a simplified model~\cite{Psms1,Psms2,Psms3,Psms4} of gluino pair production with masses in TeV range. Each of the gluinos decays to a three-body final state consisting of a pair of light quarks and an intermediate chargino. The chargino further decays to a neutralino and a W boson. The chargino mass is assumed to be the average of the masses of the gluino and the neutralino. The neutralino is considered to be the stable lightest supersymmetric particle which, in the detector, is reconstructed as substantial missing transverse energy in the final state. No b-tagged jet is expected in the final state of the targeted signal model. Therefore, in the event selection, a veto on b-tagged jets is included.\\
The SM background is dominated by $\wJets$ and $\ttJets$ events, where the isolated lepton stems from a leptonic W boson decay. Therefore, it is aligned with the mother particle. The neutrino originating from W boson decay causes an energy imbalance in the detector. The energy imbalance and the lepton together form the reconstructed W boson. In the case of a supersymmetric signal event, the existence of additional energy imbalance from neutralinos randomizes the angle between the lepton and the reconstructed W boson. Because of this distinguishing feature, we can define a separate signal and control region.\\
To enhance the sensitivity to a range of different mass configurations, multiple signal rich search regions are defined based on the number of jets, the scalar sum of all jet transverse momenta, and the scalar sum of the transverse missing momentum and the transverse lepton momentum.\\
%\null
%\thispagestyle{plain}
The contribution of SM background events in each search region is estimated using the data in the corresponding control regions. Additional sideband regions are defined in order to obtain transfer factors from signal- to control regions from data. Prior to measuring the transfer factors, the multijet events contributions are subtracted from these control regions. The prediction of multijet events is performed using measurements of lepton misidentification probabilities from data.\\
The systematic uncertainties are grouped into two categories: those that affect the background prediction and those that affect the simulated signal yields. One of the most important sources of systematic uncertainties originates from the different shape of angular distributions from dileptonic and single leptonic events. The presence of two neutrinos in $\ttbar$ events results in larger angles between the lepton and the reconstructed $W$ boson candidate than in single lepton $\ttbar$ events. The systematic uncertainty, which related to the modeling of this difference, is evaluated in data in control regions with events containing two leptons.\\
Finally, the results are interpreted in terms of the simplified models framework corresponding to the aforementioned model of gluino pair production. The results of this work is also presented in a smaller number of aggregated search regions in order to facilitate future reinterpretations.\\
%\newpage

%Organization of the thesis:
This thesis is organized as follows.

\begin{itemize}
  \item Chapter 1 starts with a brief introduction to the \acrshort{sm} of particle physics and its shortcomings. The chapter continues with discussion on supersymmetric models, as one of the most appealing theories Beyond the Standard Model (BSM). A short review of \acrshort{susy} searches at colliders is also presented.
  \item Chapter 2 gives an overview of the experimental setup used to collect data for this analysis. Prior to the description of CMS experiment, the LHC is briefly explained. Additionally, complementary simulation tools are reviewed.
  \item In Ch.~3, the reconstruction and identification of objects, such as jets,  electrons, muons, used in the CMS experiment is discussed. 
  \item The event selection is motivated and described in Ch.~4. It is followed by the definition of the search regions.
  \item Estimation of the SM background from data is explained in Ch.~5.
  \item Chapter 6 provide a description of systematic uncertainties.
  \item The final results of main search regions as well as the aggregated search regions are presented in Ch.~7. 
\end{itemize}

%\end{Introduction}

