\addtotoc{Abstract}
\chapter*{Abstract}
Supersymmetry is among the most promising theories of physics beyond the Standard Model,
but until now any direct evidence in its support is still missing.
A search for events with a single charged lepton in the final state,
coming from supersymmetric processes, is performed using proton-proton collision data
taken by the CMS experiment at the CERN LHC in the 2015 run with a center of mass energy of 13 TeV.
The integrated luminosity of the dataset corresponds to 2.3\fbinv.%, providing increased sensitivity compared to the previous 7 and 8 TeV runs.

The signal model describes gluino pair production with masses in the TeV range.
The cascade decay of each gluino involves production of 1st and 2nd generation quark jets and
a neutral stable supersymmetric particle in the final state, the lightest neutralino,
which provides a significant amount of missing transverse energy.
Exactly one charged lepton is required in the final state, which comes from the decay of one of the involved W bosons.
At the same time, multijet events are highly suppressed by this requirement.
%This provides good discrimination from QCD multijet events.
The other W boson will decay hadronically.

After applying a baseline selection to suppress the bulk of background events, the remaining events are split into a few
signal regions with different kinematic requirements.
A robust method to estimate background events from Standard Model processes using control samples in data is then introduced.
This method is validated both in simulated samples and in data.
Systematic uncertainties of the background prediction and simulated signal samples are studied.

The observation in the signal regions is in good agreement with the expectation from the Standard Model.
Therefore, exclusion limits on gluino and neutralino masses of the tested model are set.
Gluinos with masses up to 1400 GeV and neutralinos up to 850 GeV are excluded with 95\% confidence level in the considered model,
improving previous mass limits by several hundred GeVs.

%Backgrounds coming from Standard Model processes are estimated using robust methods based on control samples in data,
%which were developed and verified using simulated samples as well as data.
%The main backgrounds in the designated signal regions are coming from \ttJets and \wJets events.
%To estimate the yield of these events a transfer factor is used, which relates yields in the signal regions with yields in control regions.
%This transfer factor is measured in sideband regions in data.
%Yields of minor other background processes are directly extracted from simulation.
%Correction factors determined from simulated samples are used to compensate residual effects.
%Systematic effects and uncertainties have been studied and are applied to the prediction of background events
%as well as to expected signal yields from simulation.
%
%No significant excess of data over the expected background has been observed.
%The predicted Standard Model background very well describes the observed data.
%Therefore, exclusion limits are set.
%Gluinos with masses up to 1400 GeV and neutralinos up to 850 GeV have been excluded with 95\% confidence level in the considered model, improving previous mass limits by several hundred GeVs.
