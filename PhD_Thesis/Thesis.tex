%% ----------------------------------------------------------------
%% Thesis.tex -- MAIN FILE (the one that you compile with LaTeX)
%% ----------------------------------------------------------------

% Set up the document
\documentclass[a4paper, 11pt, twoside]{Thesis}  % Use the "Thesis" style, based on the ECS Thesis style by Steve Gunn
%\graphicspath{Figures/}  % Location of the graphics files (set up for graphics to be in PDF format)


% Include any extra LaTeX packages required
\usepackage{amsmath}
\usepackage{amssymb}

\usepackage[square, numbers, comma, sort&compress]{natbib}  % Use the "Natbib" style for the references in the Bibliography
\usepackage{verbatim}  % Needed for the "comment" environment to make LaTeX comments
\usepackage{vector}  % Allows "\bvec{}" and "\buvec{}" for "blackboard" style bold vectors in maths
\hypersetup{urlcolor=blue, colorlinks=true}  % Colours hyperlinks in blue, but this can be distracting if there are many links.
\usepackage{slashed}
\usepackage{longtable}
\usepackage{multirow}
\usepackage{fancyhdr}
\usepackage{calrsfs}
\usepackage{afterpage}
%\usepackage{fancyvrb} % Mehr Optionen für Verbatim
%\usepackage{paralist} % Für compactitem und compactenum
\usepackage{array} % Matrizen in mathematischen Formeln

%\usepackage{tocbibind}

%\usepackage{geometry} % Seitenränder und Seiteneigenschaften setzen
%\usepackage{xkeyval} % Erlaubt "Variablen" zu definieren, wird für Titelseite gebraucht

%\usepackage{tabularx, booktabs}
%\newcolumntype{Y}{>{\centering\arraybackslash}X}

\DeclareMathAlphabet{\pazocal}{OMS}{zplm}{m}{n}

\usepackage[toc,page]{appendix}
%\usepackage{feynmf}

%\usepackage[utf8]{inputenc}
\usepackage{feynmp-auto}
%\usepackage{mathtools}% http://ctan.org/pkg/mathtools
%\newcommand{\fsl}[1]{\ensuremath{\mathrlap{\!\not{\phantom{#1}}}#1}}

%
\begin{titlepage}

\begin{tabular}{ >{\centering}p{9cm} >{\centering}p{7cm} }
	\space & {\line(1,0){120}\\Unterschrift des Betreuers}
\end{tabular}

\begin{center}

% Upper part of the page
\begin{figure}[h]
	\centering
		{\includegraphics[width=0.5\textwidth]{TU_Wien_Logo.pdf}}
	 %Logo gracefully taken from http://www.tuwien.ac.at/dle/pr/publishing_web_print/corporate_design/tu_logo/
\end{figure}

\vspace{\stretch{1}}
\begin{LARGE}

\par\noindent%
	{DIPLOMARBEIT}

\vspace{\stretch{1.8}}

\textbf{Search for supersymmetry in the single lepton final state in 13 TeV pp collisions\\with the CMS experiment} \\

\end{LARGE}

\vspace{\stretch{1.8}}
\begin{large}
Ausgef\"uhrt am Atominstitut der Technischen Universit\"at Wien\\
in Verbindung mit dem Institut f\"ur Hochenergiephysik (HEPHY)\\
der \"Osterreichischen Akademie der Wissenschaften
\vspace{\stretch{0.5}}

unter der Anleitung von \\
Univ.Prof. Dipl.-Phys. Dr.rer.nat. Jochen Schieck\\
und\\
Dipl.-Ing. Dr.techn. Robert Sch\"ofbeck

\vspace{\stretch{1}}

durch \\

\vspace{\stretch{0.3}}

\LARGE
Daniel Spitzbart \\

\vspace{\stretch{0.3}}
\large
Deinhardsteingasse 29/34 \\
1160 Wien \\

\vspace{\stretch{2}}

\begin{tabular}{ >{\centering}p{8cm} >{\centering}p{8cm} }
%\centering
\line(1,0){120}\\Datum & \line(1,0){120}\\Unterschrift
\end{tabular}
\end{large}

\end{center}
\end{titlepage}


%user specific commands
\captionsetup{justification=justified,singlelinecheck=false}

\newcommand{\TFiveqqqqHM}{\ensuremath{\textrm{T5q}^{4}\textrm{WW} 1.2/0.8}\xspace}
\newcommand{\TFiveqqqqHL}{\ensuremath{\textrm{T5q}^{4}\textrm{WW} 1.5/0.1}\xspace}
\newcommand{\TFiveqqqq}{\ensuremath{\textrm{T5q}^{4}\textrm{WW}}\xspace}

\newcommand{\come}{\ensuremath{\sqrt{s}=13~\textrm{TeV}}\xspace}

\newcommand{\LT}{\ensuremath{L_{\rm T}}\xspace}
\newcommand{\ST}{\ensuremath{L_{\rm T}}\xspace}
\newcommand{\HT}{\ensuremath{H_{\rm T}}\xspace}
\newcommand{\DF}{\ensuremath{\Delta \phi}\xspace}
\newcommand{\LP}{\ensuremath{L_P}\xspace}
\newcommand{\DFWL}{\ensuremath{\Delta \phi\textrm{(W, }\ell)}\xspace}
\newcommand{\Rcs}{\ensuremath{R_{\rm CS}}\xspace}
\newcommand{\nbjet}{\ensuremath{n_{\rm b-jet}}\xspace}
\newcommand{\nbtag}{\ensuremath{n_{\rm b-tag}}\xspace}
\newcommand{\njet}{\ensuremath{n_{\rm jet}}\xspace}
\newcommand{\njetSR}{\ensuremath{n_\textrm{jet}^{\textrm{SR}}}\xspace}
\newcommand{\gluino}{\ensuremath{\widetilde{g}}\xspace} % gluino
\newcommand{\nino}{\ensuremath{\widetilde{\chi}}\xspace} % neutralino
\newcommand{\ninozero}{\ensuremath{\widetilde{\chi}^0}\xspace} % neutralino with superscript 0
\newcommand{\ninoone}{\ensuremath{\widetilde{\chi}^0_1}\xspace} % neutralino with superscript 0 and subscript 1
\newcommand{\ninotwo}{\ensuremath{\widetilde{\chi}^0_2}\xspace} % neutralino with superscript 0 and subscript 2
\newcommand{\ninothree}{\ensuremath{\widetilde{\chi}^0_3}\xspace} % neutralino with superscript 0 and subscript 3
\newcommand{\ninofour}{\ensuremath{\widetilde{\chi}^0_4}\xspace} % neutralino with superscript 0 and subscript 4
\newcommand{\chipm}{\ensuremath{\widetilde{\chi}^{\pm}}\xspace} % chargino with superscript pm and subscript 1
\newcommand{\chipmone}{\ensuremath{\widetilde{\chi}^{\pm}_1}\xspace} % chargino with superscript pm and subscript 1
\newcommand{\chipone}{\ensuremath{\widetilde{\chi}^{+}_1}\xspace} % chargino with superscript p and subscript 1
\newcommand{\chipmtwo}{\ensuremath{\widetilde{\chi}^{\pm}_2}\xspace} % chargino with superscript pm and subscript 1


\newcommand{\mgl}{\ensuremath{m_{\widetilde{g}}}\xspace} % gluino
\newcommand{\mnzero}{\ensuremath{m_{\widetilde{\chi}^0}}\xspace} % neutralino with superscript 0
\newcommand{\mnone}{\ensuremath{m_{\widetilde{\chi}^0_1}}\xspace} % neutralino with superscript 0
\newcommand{\mntwo}{\ensuremath{m_{\widetilde{\chi}^0_2}}\xspace} % neutralino with superscript 0
\newcommand{\mcone}{\ensuremath{m_{\widetilde{\chi}^{\pm}_1}}\xspace} % chargino with superscript pm and subscript 1


\newcommand{\fbinv}{\ensuremath{{\textrm{fb}^{-1}}}\xspace}
\newcommand{\sqrts}{\ensuremath{\sqrt{s}}\xspace}
\newcommand{\btag}{\ensuremath{{\rm b-tag}}\xspace}
\newcommand{\zeroTag}{\ensuremath{0\textrm{-tag}}\xspace}
\newcommand{\oneTag}{\ensuremath{1\textrm{-tag}}\xspace}
\newcommand{\twoTag}{\ensuremath{\ge\hspace{-0.08em}2\textrm{-tag}}\xspace}
\newcommand{\onePlusTag}{\ensuremath{\ge\hspace{-0.08em}1\textrm{-tag}}\xspace}
\newcommand{\Rcscorr}{\ensuremath{R_{\textrm{CS}}^{\textrm{corr.}}}}
\newcommand{\Rcsw}{\ensuremath{R_{\textrm{CS}}^{W}}}
\newcommand{\Rcstt}{\ensuremath{R_{\textrm{CS}}^{t\bar{t}}}}

\newcommand{\kappab}{\ensuremath{\kappa_{b}}\xspace}
\newcommand{\kappaw}{\ensuremath{\kappa_{W}}\xspace}
\newcommand{\kappatt}{\ensuremath{\kappa_{t\bar{t}}}\xspace}

\newcommand{\wJets}{W+jets\xspace}
\newcommand{\ttJets}{\ensuremath{t\bar{t}+\textrm{jets}}\xspace}
\newcommand{\ttWJets}{\ensuremath{t\bar{t}W+\textrm{jets}}\xspace}
\newcommand{\singleTop}{\ensuremath{t/\bar{t}}\xspace}
\newcommand{\TTVH}{\ensuremath{t\bar{t}W}\xspace}
\newcommand{\DY}{\ensuremath{\textrm{Drell-Yan}}\xspace}


\newcommand{\wpJets}{W+jets\xspace}
\newcommand{\wmJets}{W+jets\xspace}
\newcommand{\yFit}{\ensuremath{y^{\textrm{fit}}}}
\newcommand{\yPred}{\ensuremath{y^{\textrm{pred.}}}}
\newcommand{\yMC}{\ensuremath{y^{\textrm{MC}}}}
\newcommand{\MPF}{\ensuremath{\textrm{MPF}}}
\newcommand{\MET}{\ensuremath{\slashed{E}_T}\xspace}
\newcommand{\METvec}{\ensuremath{\vec{\slashed{E}}_T}\xspace}
\newcommand{\fakeMET}{\ensuremath{\slashed{E}_T^{\textrm{fake}}}\xspace}
\newcommand{\genMET}{\ensuremath{\slashed{E}_T^{\textrm{gen}}}\xspace}
\newcommand{\recoMET}{\ensuremath{\slashed{E}_T^{\textrm{reco}}}\xspace}
\newcommand{\trueLT}{\ensuremath{L_T^{\textrm{true}}}\xspace}


\newcommand{\pt}{\ensuremath{p_T}\xspace}
\newcommand{\ptvec}{\ensuremath{\vec{p}_T}\xspace}
\newcommand{\ttbar}{\ensuremath{t\bar{t}}\xspace}

\hypersetup{ % Setzt einige Werte die in den Eigenschaften des PDF gespeichert sind.
	pdfauthor = {Daniel Spitzbart},
	pdftitle = {Search for supersymmetry in the single lepton final state in 13 TeV pp collisions with the CMS experiment},
	pdfsubject = {Diplomarbeit Physik},
	pdfkeywords = {diplom, master},
	pdfdisplaydoctitle = true,
	colorlinks = false, % Für Druck auf "false" setzen!
}


%% ----------------------------------------------------------------
\begin{document}
\frontmatter      % Begin Roman style (i, ii, iii, iv...) page numbering


\begin{titlepage}

\begin{tabular}{ >{\centering}p{9cm} >{\centering}p{7cm} }
	\space & {\line(1,0){120}\\Unterschrift des Betreuers}
\end{tabular}

\begin{center}

% Upper part of the page
\begin{figure}[h]
	\centering
		{\includegraphics[width=0.5\textwidth]{TU_Wien_Logo.pdf}}
	 %Logo gracefully taken from http://www.tuwien.ac.at/dle/pr/publishing_web_print/corporate_design/tu_logo/
\end{figure}

\vspace{\stretch{1}}
\begin{LARGE}

\par\noindent%
	{DIPLOMARBEIT}

\vspace{\stretch{1.8}}

\textbf{Search for supersymmetry in the single lepton final state in 13 TeV pp collisions\\with the CMS experiment} \\

\end{LARGE}

\vspace{\stretch{1.8}}
\begin{large}
Ausgef\"uhrt am Atominstitut der Technischen Universit\"at Wien\\
in Verbindung mit dem Institut f\"ur Hochenergiephysik (HEPHY)\\
der \"Osterreichischen Akademie der Wissenschaften
\vspace{\stretch{0.5}}

unter der Anleitung von \\
Univ.Prof. Dipl.-Phys. Dr.rer.nat. Jochen Schieck\\
und\\
Dipl.-Ing. Dr.techn. Robert Sch\"ofbeck

\vspace{\stretch{1}}

durch \\

\vspace{\stretch{0.3}}

\LARGE
Daniel Spitzbart \\

\vspace{\stretch{0.3}}
\large
Deinhardsteingasse 29/34 \\
1160 Wien \\

\vspace{\stretch{2}}

\begin{tabular}{ >{\centering}p{8cm} >{\centering}p{8cm} }
%\centering
\line(1,0){120}\\Datum & \line(1,0){120}\\Unterschrift
\end{tabular}
\end{large}

\end{center}
\end{titlepage}



%% Set up the Title Page
%\title  {Search for supersymmetry in the single lepton final state in 13 TeV pp collisions with the CMS experiment}
%\authors  {\texorpdfstring
%            {\href{your web site or email address}{Author Name}}
%            {Daniel Spitzbart}
%            }
%\addresses  {\groupname\\\deptname\\\univname}  % Do not change this here, instead these must be set in the "Thesis.cls" file, please look through it instead
%\date       {\today}
%\subject    {}
%\keywords   {}
%
%\maketitle
%%% ----------------------------------------------------------------

\setstretch{1.3}  % It is better to have smaller font and larger line spacing than the other way round

% Define the page headers using the FancyHdr package and set up for one-sided printing
\fancyhead{}  % Clears all page headers and footers
\rhead{\thepage}  % Sets the right side header to show the page number
\lhead{}  % Clears the left side page header

\pagestyle{empty}  % Finally, use the "fancy" page style to implement the FancyHdr headers

\afterpage{\null\newpage}

%% ----------------------------------------------------------------
% Declaration Page required for the Thesis, your institution may give you a different text to place here
%\Declaration{
%
%\addtocontents{toc}{\vspace{1em}}  % Add a gap in the Contents, for aesthetics
%
%I, AUTHOR NAME, declare that this thesis titled, `THESIS TITLE' and the work presented in it are my own. I confirm that:
%
%\begin{itemize}
%\item[\tiny{$\blacksquare$}] This work was done wholly or mainly while in candidature for a research degree at this University.
%
%\item[\tiny{$\blacksquare$}] Where any part of this thesis has previously been submitted for a degree or any other qualification at this University or any other institution, this has been clearly stated.
%
%\item[\tiny{$\blacksquare$}] Where I have consulted the published work of others, this is always clearly attributed.
%
%\item[\tiny{$\blacksquare$}] Where I have quoted from the work of others, the source is always given. With the exception of such quotations, this thesis is entirely my own work.
%
%\item[\tiny{$\blacksquare$}] I have acknowledged all main sources of help.
%
%\item[\tiny{$\blacksquare$}] Where the thesis is based on work done by myself jointly with others, I have made clear exactly what was done by others and what I have contributed myself.
%\\
%\end{itemize}
%
%
%Signed:\\
%\rule[1em]{25em}{0.5pt}  % This prints a line for the signature
%
%Date:\\
%\rule[1em]{25em}{0.5pt}  % This prints a line to write the date
%}
\clearpage  % Declaration ended, now start a new page

%% ----------------------------------------------------------------
% The "Funny Quote Page"
\pagestyle{empty}  % No headers or footers for the following pages

\null\vfill
% Now comes the "Funny Quote", written in italics
\textit{``Look closely. The beautiful may be small.''}

\begin{flushright}
Immanuel Kant
\end{flushright}

\vfill\vfill\vfill\vfill\vfill\vfill\null
\clearpage  % Funny Quote page ended, start a new page
%% ----------------------------------------------------------------

% The Abstract Page

\begin{abstract}
%The Standard Model of particle physics is like an old family car:  likable but also with problems, like the hierarchy and the lack of explanation of Dark Matter. Many extensions of the Standard Model provide solutions to these problems, and Supersymmetry seems to be one of the most promising ones. 
%A search for Supersymmetry in events with a single electron or muon is performed on proton-proton collisions at a center-of-mass energy of 13 TeV. The data were recorded by the CMS experiment during Run 2 of the LHC, corresponding to an integrated luminosity of 36.5 fb-1. 
%The analysis is designed to look for signatures of the two different decays of pair-produced gluinos, superpartners of Standard Model gluons. In one of them each gluino decays to top quarks and a neutralino via a three-body decay. In the other one, each gluino decays to two light quarks and an intermediate chargino, with the latter decaying to a W boson and a neutralino. In these models, the neutralino is considered to be the stable lightest supersymmetric particle, or LSP. Hence, It is a strong candidate of Dark Matter.
%The main search variable of the analysis is the azimuthal angle between the lepton and four-vector sum of the missing energy and lepton. The angle for leading background processes tend towards low values while the expected signal events do not show dependence, due to the large missing transverse energy contribution from LSP. Thus, the region with high (low) values of this angle is chosen to be signal (control) region. To further increase the sensitivity several signal rich search regions are defined, based on the number of (b)jets, the scalar sum of all jet transverse momenta, and the scalar sum of the transverse missing momentum and transverse lepton momentum. The Standard Model background is estimated with a data-driven approach using control regions where no signal contribution is expected. Low jet multiplicity sidebands are used to obtain signal to control region transfer factor. 
%Since no significant deviation from the predicted Standard Model background is observed, exclusion limits on gluino and neutralino masses are obtained.
In this thesis, an inclusive search for supersymmetry is presented. The search is performed in events containing a single lepton, multiple jets requiring none of them coming from b quarks, and missing transverse energy in the final state. The proton-proton collision data were recorded by the CMS experiment during Run 2 of the LHC at a center-of-mass energy of 13 TeV. The analyzed data corresponds to a total integrated luminosity of 35.9 fb$^{-1}$. The search uses delta phi, the azimuthal angle between the lepton and four-vector sum of the missing energy and lepton, as a powerful discriminating variable to distinguish between background and signal. Additionally, multiple exclusive search regions are defined in different kinematic observables to enhance sensitivity to a range of different mass scenarios. The latest results in this clean event topology interpreted in the context of simplified models.

\end{abstract}

%\clearpage  % Abstract ended, start a new page

\input{kurzfassung}
%\clearpage  % Abstract ended, start a new page
%% ----------------------------------------------------------------

\setstretch{1.3}  % Reset the line-spacing to 1.3 for body text (if it has changed)

% The Acknowledgements page, for thanking everyone
%\acknowledgements{
%\addtocontents{toc}{\vspace{1em}}  % Add a gap in the Contents, for aesthetics
%
%The acknowledgements and the people to thank go here, don't forget to include your project advisor\ldots
%
%}
\input{acknowledgements}
%\clearpage  % End of the Acknowledgements
%% ----------------------------------------------------------------

\pagestyle{fancy}  %The page style headers have been "empty" all this time, now use the "fancy" headers as defined before to bring them back


%% ----------------------------------------------------------------
\lhead{\emph{Contents}}  % Set the left side page header to "Contents"
\tableofcontents  % Write out the Table of Contents



%% ----------------------------------------------------------------
%\setstretch{1.5}  % Set the line spacing to 1.5, this makes the following tables easier to read
%\clearpage  % Start a new page
%\lhead{\emph{Abbreviations}}  % Set the left side page header to "Abbreviations"
%\listofsymbols{ll}  % Include a list of Abbreviations (a table of two columns)
%{
%% \textbf{Acronym} & \textbf{W}hat (it) \textbf{S}tands \textbf{F}or \\
%\textbf{LAH} & \textbf{L}ist \textbf{A}bbreviations \textbf{H}ere \\
%
%}

%% ----------------------------------------------------------------
%\clearpage  % Start a new page
%\lhead{\emph{Physical Constants}}  % Set the left side page header to "Physical Constants"
%\listofconstants{lrcl}  % Include a list of Physical Constants (a four column table)
%{
%% Constant Name & Symbol & = & Constant Value (with units) \\
%Speed of Light & $c$ & $=$ & $2.997\ 924\ 58\times10^{8}\ \mbox{ms}^{-\mbox{s}}$ (exact)\\
%
%}

%% ----------------------------------------------------------------
%\clearpage  %Start a new page
%\lhead{\emph{Symbols}}  % Set the left side page header to "Symbols"
%\listofnomenclature{lll}  % Include a list of Symbols (a three column table)
%{
%% symbol & name & unit \\
%$a$ & distance & m \\
%$P$ & power & W (Js$^{-1}$) \\
%& & \\ % Gap to separate the Roman symbols from the Greek
%$\omega$ & angular frequency & rads$^{-1}$ \\
%}
%% ----------------------------------------------------------------
% End of the pre-able, contents and lists of things
% Begin the Dedication page

\setstretch{1.3}  % Return the line spacing back to 1.3

\pagestyle{empty}  % Page style needs to be empty for this page
\dedicatory{To my family.}% my parents and grandparents, whom I owe everything.}

\addtocontents{toc}{\vspace{2em}}  % Add a gap in the Contents, for aesthetics


%% ----------------------------------------------------------------
\mainmatter	  % Begin normal, numeric (1,2,3...) page numbering
%\lhead{}
\fancyhead[LE,RO]{\thepage}
\fancyhead[RE,LO]{\leftmark}
\pagestyle{fancy}  % Return the page headers back to the "fancy" style

% Include the chapters of the thesis, as separate files
% Just uncomment the lines as you write the chapters

%\chapter{Searches for SUSY with the $\textrm{R}_{\textrm{CS}}$ Method}
%\input{Chapters/Intro} %Introduction

\input{Chapters/Chapter3} % SM & SUSY

\input{Chapters/Chapter1} % LHC

\input{Chapters/Chapter2} % CMS

\input{Chapters/Chapter4/reconstruction}
\input{Chapters/Chapter4}
\input{Chapters/Chapter4/validation}

%\input{Chapters/Chapter4} % Search

\input{Chapters/Chapter5} % Results

\input{Chapters/Chapter6} % Conclusion




%% ----------------------------------------------------------------
% Now begin the Appendices, including them as separate files

\cleardoublepage

\addtocontents{toc}{\vspace{2em}} % Add a gap in the Contents, for aesthetics

\begin{appendices}

\input{Appendices/AppendixA}	% Appendix Title

%\input{Appendices/AppendixB} % Appendix Title

%\input{Appendices/AppendixC} % Appendix Title
\end{appendices}

%\pagestyle{fancy}  %The page style headers have been "empty" all this time, now use the "fancy" headers as defined before to bring them back

\cleardoublepage

\addtocontents{toc}{\vspace{2em}}  % Add a gap in the Contents, for aesthetics

%% ----------------------------------------------------------------
\lhead{\emph{List of Figures}}  % Set the left side page header to "List if Figures"
\listoffigures  % Write out the List of Figures

%\addtocontents{toc}{\vspace{2em}}

%% ----------------------------------------------------------------
\lhead{\emph{List of Tables}}  % Set the left side page header to "List of Tables"
\listoftables  % Write out the List of Tables

\addtocontents{toc}{\vspace{2em}}  % Add a gap in the Contents, for aesthetics

\backmatter

%% ----------------------------------------------------------------
\label{Bibliography}
\lhead{\emph{Bibliography}}  % Change the left side page header to "Bibliography"
\bibliographystyle{utphys}  % Use the "unsrtnat" BibTeX style for formatting the Bibliography
\bibliography{Bibliography}  % The references (bibliography) information are stored in the file named "Bibliography.bib"

\end{document}  % The End
%% ----------------------------------------------------------------
