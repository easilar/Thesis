\begin{abstract}
%The Standard Model of particle physics is like an old family car:  likable but also with problems, like the hierarchy and the lack of explanation of Dark Matter. Many extensions of the Standard Model provide solutions to these problems, and Supersymmetry seems to be one of the most promising ones. 
%A search for Supersymmetry in events with a single electron or muon is performed on proton-proton collisions at a center-of-mass energy of 13 TeV. The data were recorded by the CMS experiment during Run 2 of the LHC, corresponding to an integrated luminosity of 36.5 fb-1. 
%The analysis is designed to look for signatures of the two different decays of pair-produced gluinos, superpartners of Standard Model gluons. In one of them each gluino decays to top quarks and a neutralino via a three-body decay. In the other one, each gluino decays to two light quarks and an intermediate chargino, with the latter decaying to a W boson and a neutralino. In these models, the neutralino is considered to be the stable lightest supersymmetric particle, or LSP. Hence, It is a strong candidate of Dark Matter.
%The main search variable of the analysis is the azimuthal angle between the lepton and four-vector sum of the missing energy and lepton. The angle for leading background processes tend towards low values while the expected signal events do not show dependence, due to the large missing transverse energy contribution from LSP. Thus, the region with high (low) values of this angle is chosen to be signal (control) region. To further increase the sensitivity several signal rich search regions are defined, based on the number of (b)jets, the scalar sum of all jet transverse momenta, and the scalar sum of the transverse missing momentum and transverse lepton momentum. The Standard Model background is estimated with a data-driven approach using control regions where no signal contribution is expected. Low jet multiplicity sidebands are used to obtain signal to control region transfer factor. 
%Since no significant deviation from the predicted Standard Model background is observed, exclusion limits on gluino and neutralino masses are obtained.
In this thesis, an inclusive search for supersymmetry is presented. The search is performed in events containing a single lepton, multiple jets requiring none of them coming from b quarks, and missing transverse energy in the final state. The proton-proton collision data were recorded by the CMS experiment during Run 2 of the LHC at a center-of-mass energy of 13 TeV. The analyzed data corresponds to a total integrated luminosity of 35.9 fb$^{-1}$. The search uses delta phi, the azimuthal angle between the lepton and four-vector sum of the missing energy and lepton, as a powerful discriminating variable to distinguish between background and signal. Additionally, multiple exclusive search regions are defined in different kinematic observables to enhance sensitivity to a range of different mass scenarios. The latest results in this clean event topology interpreted in the context of simplified models.

\end{abstract}
