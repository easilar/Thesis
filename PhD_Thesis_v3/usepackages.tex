% Set borders
\usepackage[left=1.8in, right=1.2in, top=1.5in, bottom=0.95in, includefoot, headheight=13.6pt]{geometry}
%\usepackage[right=1.5in]{geometry}

% Fix paragraph indentation to no indent, and skip one line instead. 
\parindent 0pt
\parskip 1ex

% Caption text style
\usepackage{caption}
\DeclareCaptionStyle{normal}
{
  font={normalsize}, textfont={normal}, labelfont={bf}, margin=20pt, figureposition=bottom, tableposition=top
}

\captionsetup{style=normal}
\setlength{\abovecaptionskip}{0pt}

% Fix hyperrefs
\usepackage[figure,table]{hypcap} % Correct a problem with hyperref
\makeatletter
\newcommand\org@hypertarget{}
\let\org@hypertarget\hypertarget
\renewcommand\hypertarget[2]{%
\Hy@raisedlink{\org@hypertarget{#1}{}}#2%
} \makeatother

% Disable single lines at the start of a paragraph (Schusterjungen)
\clubpenalty = 10000
% Disable single lines at the end of a paragraph (Hurenkinder)
\widowpenalty = 10000 \displaywidowpenalty = 1000

% This is so you can use utf8 encoded characters in your document
\usepackage[utf8]{inputenc}

% This is to get the correct font encoding
\usepackage[T1]{fontenc} 

% This loads hyphenations for german and english (autmatic adding of
% \- to long words
\usepackage[german,english]{babel}

% This is so we can make the appendix section begin and end
\usepackage[toc,page]{appendix}

% Use graphics with pdftex option
\usepackage[pdftex]{graphicx}

% font color
\usepackage{color}

% Use to get multiple figures in one figure float
\usepackage{caption}
\usepackage{subcaption}
\usepackage{floatrow}
\floatsetup[table]{style=plaintop} %to put table caption back on top
%\usepackage{subfigure}
\usepackage{sidecap}
\usepackage{float}
\usepackage{tikz}
\restylefloat{figure}
%\usepackage[section] {placeins}

% Better continued figures
\usepackage[countmax]{subfloat}

% Better tables
\usepackage{tabularx}
\usepackage{longtable}
\usepackage{ltxtable}

% This is useful to make a table cell span more than one row, much like multicol
\usepackage{multirow}

% This is useful for typesetting particle names, see
% http://www.tex.ac.uk/tex-archive/help/Catalogue/entries/hepparticles.html
\usepackage{hepparticles}

% To nice print acronyms
% \ac{atlas}
\usepackage[printonlyused,withpage]{acronym}

% Typeset web links proper
\usepackage{url} 

%math
\usepackage{amsmath}
\usepackage{amssymb}
\usepackage{braket}

%rotating stuff
\usepackage{rotating}

%table
\usepackage{longtable}
\usepackage{lscape}

\usepackage{afterpage}

%enumeration
\usepackage{enumitem}

\usepackage{blindtext} % for dummy text

\usepackage{ifthen}