\chapter{Introduction}
The new energy level of LHC allowed experimentalists to perform various searches.
These searches have probbed a broad variety of possible theories that can explain how the universe is working, i.e
from how objects are falling to how quarks compose protons.
One of the most promissing ones was the Standard model (SM) of particle physics. As a result of comprehensive searches performed by the CMS and ATLAS experiments at LHC the discovery of Higgs Boson in 2012 completed the last missing part of the theory.
Up to now, the SM has succesfully explained the physics observed in LHC.
However, SM has several shortcomings.
The lack of explanation of gravity and dark matter are some of the most important ones.
 Moreover, the way SM is introducing the Higgs boson mass includes an unnatural fine tuning, so called hierarchy problem. 
Therefore, the journey of particle physics has not ended. 
% Nowadays some of the most favourable searches to understand what is beyond the SM are searches for supe searches.  
There are many extensions of the SM. One of the most favourable ones is Supersymmetry (SUSY). 


\newpage
Organisation of the thesis:

> itemize the chapters


