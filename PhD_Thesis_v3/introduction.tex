\begin{Introduction}
%\chapter{Introduction}
FIXME: To be written in a nicer way, at the end...
%The new energy level of LHC allowed experimentalists to perform various searches.
%These searches have probbed a broad variety of possible theories that can explain how the universe is working, i.e
%from how objects are falling to how quarks compose protons.
%One of the most promissing ones was the Standard model (SM) of particle physics. As a result of comprehensive searches performed by the CMS and ATLAS experiments at LHC the discovery of Higgs Boson in 2012 completed the last missing part of the theory.
%Up to now, the SM has succesfully explained the physics observed in LHC.
%However, SM has several shortcomings.
%The lack of explanation of gravity and dark matter are some of the most important ones.
 %Moreover, the way SM is introducing the Higgs boson mass includes an unnatural fine tuning, so called hierarchy problem. 
%Therefore, the journey of particle physics has not ended. 
% Nowadays some of the most favourable searches to understand what is beyond the SM are searches for supe searches.  
%There are many extensions of the SM. One of the most favourable ones is Supersymmetry (SUSY). 
%%%%%%%%%%%
%The new energy level of LHC allowed experimentalists to perform various searches.
%These searches have probed a broad variety of theories that can possibly explain how the universe is working, i.e from how objects are falling to how protons compose of quarks and gluons.
%One of the most promising ones was the Standard model (SM) of particle physics. As a result of comprehensive searches performed by the CMS and ATLAS experiments at LHC, the discovery of Higgs Boson in 2012 completed the last missing part of the theory.
%Even though the SM has been tested many times until now, It predicts the quantifiable quantities of interactions between the properties of subatomic particles with great precision. Hence, the SM has successfully explained the physics observed in LHC.
%However, SM has several shortcomings.
%The lack of explanation of gravity and dark matter are some of the most important ones.
 %Moreover, the way SM is introducing the Higgs boson mass includes an unnatural fine-tuning, so called hierarchy problem. 
%Therefore, the particle physics continues to its journey to find solutions to these problems. 
%One of the most favorable ones was proposed as an extension of space-time symmetry associates the fermions and the bosons. The theory is meaningfully named as Supersymmetry (SUSY). 
%It is successfully solving the two of the problems mentioned above. It is providing a weakly interacting particle candidate for dark matter with involving a stable neutral supersymmetric particle. Moreover, the cancellations between bosonic and fermionic states result in a reduced size of quantum corrections. Hence, SUSY includes a solution to the hierarchy problem by itself.

%This thesis focuses on a search for Supersymmetry in events with a single lepton performed on proton-proton collisions at a center-of-mass energy of 13 TeV. The data is recorded by the CMS experiment during Run 2 of the LHC.
%A R-parity conserving simplified model describing the decay of pair-produced gluinos, superpartners of the Standard Model gluons is studied. Each gluino decays to two light quarks and an intermediate chargino, with the latter decaying to a W boson and a neutralino. In this model, the neutralino is considered to be the stable lightest supersymmetric particle. Hence, It is a sensible candidate of dark matter.
%The main search variable is the azimuthal angle between the lepton and four-vector sum of the missing energy and lepton. The angle for leading background processes tend towards low values while the expected signal events do not show dependence, due to the large missing transverse energy contribution from LSP. Thus, the region with high (low) values of this angle is chosen to be signal (control) region. To further increase the sensitivity several signal rich search regions are defined, based on the number of (b)jets, the scalar sum of all jet transverse momenta, and the scalar sum of the transverse missing momentum and transverse lepton momentum. The Standard Model background is estimated with a data-driven approach using control regions where no signal contribution is expected. Low jet multiplicity sidebands are used to obtain signal to control region transfer factor.


\newpage
%Organisation of the thesis:
This thesis is organized as follows.

> itemize the chapters
\end{Introduction}

