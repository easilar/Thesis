\chapter{Conclusion}
The Standard Model (SM) of the particle physics continues to be a successful theory in explaining the physical phenomenon observed in LHC. However, the SM fails to elucidate several experimental facts that give rise to questions regarding the origin of our universe such as: What is Dark Matter and how gravity can be included?\\
The supersymmetric extension of the SM (SUSY) is one of the most appealing beyond the SM (BSM) theories that provide solutions to some of these questions. SUSY relates the fermions to bosonic superpartners and bosons to fermionic superpartners. This characteristic allows SUSY to reduce the divergent corrections on the Higgs mass.  Furthermore, the R-parity conserved SUSY models provide a cold-dark matter candidate in the form of lightest SUSY particles. \\
In this thesis, a search for a simplified SUSY model is performed in events with single lepton final states, multiple jets and none of them zero b-tagged. In the targeted model, each of the pair produced gluinos decays to neutralino, which is considered to be stable lightest supersymmetric particle, together with several light quark jets and one electron or muon. The intermediate chargino mass in this decay chain is taken to be half way between gluino and neutralino. The two neutralinos and a neutrino coming from leptonic W boson decay combine a large $\MET$. Moreover, the high multiplicity of jets leads to large hadronic activity. Therefore the model has high sensitivity in the tails of the kinematic distributions.  \\
The main search variable of the analysis is the azimuthal angle between the lepton and reconstructed $W$ boson, $\Delta\phi(W,\ell)$. The leading background processes tend towards low values of the angle while the expected signal events show a flat distribution, due to the large missing transverse energy contribution from LSP. Thus, the region with high (low) values of this angle is chosen to be signal (control) region. To further increase the sensitivity several signal rich search regions are defined, based on the number of jets, the hadronic scale ($\HT$), and the leptonic scale ($\LT$).\\
The Standard Model background is predicted with a robust data-driven approach, so-called $\Rcs$. The method uses the control regions to estimate the normalization in signal regions and low jet multiplicity sidebands to obtain signal to control region transfer factor. The main SM backgrounds $\wJets$ and $\ttJets$ are predicted separately. A separate estimation method is developed for QCD multijets events.  This method uses the polarization information of leptons from W boson decays which manifests itself in the $\rm L_P$ distributions. The predicted QCD multijets event counts are subtracted from the control regions. All of the methods are validated in simulation and data. \\
Various sources of systematic uncertainties on the background prediction are computed. The largest contributions are from possible residual dependencies of $\Rcs$ on the jet multiplicity and potential mismodeling of the dileptonic fraction of $\ttbar$ events. Additionally, the systematic uncertainties coming from the simulated signal events are studied and their effects are propagated. In this case, the largest uncertainties arise from jet energy correction uncertainties and potential mismodeling of pile up. Other uncertainties are at the 5\% level or lower.\\
Finally, the prediction is compared to data recorded by the CMS experiment during the 2016 run of proton-proton collisons of LHC at 13 TeV center of mass energy. The dataset corresponds to an integrated luminosity of 35.9 fb$^{-1}$. No deviation from the SM background is observed. Therefore, upper limits on gluino production cross section are obtained in the context of the investigated simplified model. 95\% C.L. are calculated with the asymptotic formulae using the $\rm CL_s$ criterion. As a result, gluino masses below 1.9 TeV are excluded for neutralino masses below 300 GeV.\\
After the latest with 35.9 inverse femtobarns of 13 TeV data still results do not favor the SUSY models. The minimal supersymmetric models become more and more constraint. As the limits on the gluino mass approach to two TeV scales, the natural SUSY become more restricted. Given the variety of possibilities, completely excluding SUSY is difficult. The SUSY searches can be expended towards the direction of long-lived particles with displaced vertex analyses, while there has been still a place for natural SUSY in the compressed regions. \\
Although no significant deviation from the Standard Model has been observed, the journey of particle physics has not completed yet. The unanswered questions from the observation of astrophysical phenomena remains as they are as long as an evidence of a more universal model is not observed. \\
Over the last decades, the experimental searches have been performed under the guidance of theoretical physics methods. The success of this scientific path is proven with the Higgs boson discovery and the gravitational waves.
It is also possible that a sudden unexpected discovery can illuminate the theoretical physics path. 